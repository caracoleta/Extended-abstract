%%%%%%%%%%%%%%%%%%%%%%%%%%%%%%%%%%%%%%%%%%%%%%%%%%%%%%%%%%%%%%%%%%%%%%
%     File: ExtendedAbstract_backg.tex                               %
%     Tex Master: ExtendedAbstract.tex                               %
%                                                                    %
%     Author: Andre Calado Marta                                     %
%     Last modified : 27 Dez 2011                                    %
%%%%%%%%%%%%%%%%%%%%%%%%%%%%%%%%%%%%%%%%%%%%%%%%%%%%%%%%%%%%%%%%%%%%%%
% A Theory section should extend, not repeat, the background to the
% article already dealt with in the Introduction and lay the
% foundation for further work.
%%%%%%%%%%%%%%%%%%%%%%%%%%%%%%%%%%%%%%%%%%%%%%%%%%%%%%%%%%%%%%%%%%%%%%

\section{Literature review}
\label{sec:backg}
In this section a review of the previous literature on airport efficiency is conducted.

Spain, having one of the largest airport operator (Aena), is one of the most frequently analyzed countries in the literature. Thus, a special focus is given to works analyzing Spanish airports, representing 10 of the 22 studies relevant studies reviewed.

Examining the literature, it was possible to identify that DEA is the most common methodology used to assess airport efficiency, being used in 19 of the 22 studies reviewed. Nevertheless, SFA has also been applied in relevant Iberian studies, such as \cite{barros2008}, \cite{tovar2010}, \cite{martin2011}. The DEA's non-parametric nature, meaning it does not require a predefined functional form for the production function, makes it the preferred method. 

From the 19 DEA studies, 11 combined the first-stage DEA with a second-stage analysis to identify exogenous factors influencing airport efficiency. Some articles applied the Tobit regression (\cite{coto-millan2014}, \cite{fragoudaki2016}, \cite{coto-millan2016}) but the truncated regression proposed by \cite{simar2007} was deemed more appropriate and was used in 7 of the 11 studies (\cite{barrosdieke2008}, \cite{barros2008b}, \cite{tsekeris2011}, \cite{chang2013}, \cite{adler2013}, \cite{fernandez2022}, \cite{cifuentes-faura2023}).
Also, the static DEA scores were complemented with a dynamic analysis using the Malmquist Productivity Index (MPI) in 4 of the 13 studies that analyzed more than one year (\cite{fung2008}, \cite{tovar2010}, \cite{coto-millan2014}, \cite{inglada2018}).

Regarding the input variables used, some studies used only financial inputs, such as capital or operating costs (\cite{martin2001}, \cite{barros2008}, \cite{barrosdieke2008}, \cite{coto-millan2014}, \cite{coto-millan2016}, \cite{inglada2018}), but due to the common lack of financial data, most studies used physical inputs, such as the runway area/length or apron area, that are more related to the areonautical capacity of the airport (\cite{lin2006}, \cite{barros2008b}, \cite{tsekeris2011}, \cite{lozano2013}, \cite{fragoudaki2016}, \cite{gutierrez2016}) or logistic inputs, such as the number of boarding gates, number of employees or the terminal area.
Other variables such as the airport area and number of baggage belts, check-in counters, airlines and routes were also identified but less frequently employed (\cite{lozano2013}, \cite{gutierrez2016}, \cite{cifuentes-faura2023}). 

For output variables, the literature clearly converges
on passenger volume, aircraft movements, and cargo as the most common output variables, with 18 studies presenting them in their analysis.

A review of the explanatory variables used in the second-stage analysis was also performed. The selection of these variables depends on the scope of each study, but common variables include location, working load unit, military presence and hub status (\cite{barrosdieke2008}, \cite{barros2008b}, \cite{tsekeris2011}, \cite{adler2013}, \cite{fragoudaki2016}).

Accordingly, the present study will employ a two-stage DEA approach with a truncated regression, complemented by a Malmquist Productivity Index analysis, to evaluate the efficiency of 41 Iberian airports, from 2016 to 2023, and identify exogenous factors influencing their performance. 
This study contributes to the Iberian airport efficiency literature, by analyzing for the first time Portuguese and Spanish airports simultaneously. Additionally, it will be one of the few studies incorporating data after the partial privatization of Aena (Spanish airport operator) in 2014. 