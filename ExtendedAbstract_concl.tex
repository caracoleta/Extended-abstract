%%%%%%%%%%%%%%%%%%%%%%%%%%%%%%%%%%%%%%%%%%%%%%%%%%%%%%%%%%%%%%%%%%%%%%
%     File: ExtendedAbstract_concl.tex                               %
%     Tex Master: ExtendedAbstract.tex                               %
%                                                                    %
%     Author: Andre Calado Marta                                     %
%     Last modified : 27 Dez 2011                                    %
%%%%%%%%%%%%%%%%%%%%%%%%%%%%%%%%%%%%%%%%%%%%%%%%%%%%%%%%%%%%%%%%%%%%%%
% The main conclusions of the study presented in short form.
%%%%%%%%%%%%%%%%%%%%%%%%%%%%%%%%%%%%%%%%%%%%%%%%%%%%%%%%%%%%%%%%%%%%%%

\section{Conclusions}
\label{sec:concl}
In this section, the main achievements of this work are summarized, followed by public policy reflections, and finally the main limitations and suggestions for future work are presented.
\subsection{Achievements}

The first-stage DEA analysis revealed that, on average, Iberian airports present relatively low effi-
ciency values in the considered period, with an average bootstrapped VRS efficiency score of 0.480, which increases to 0.522 when excluding the years affected by the COVID-19 pandemic (2020 and
2021). The results also indicated that most airports are not operating at optimal scale, with a significant
number presenting increasing returns to scale, confirming that several airports operate with overcapacity. Lastly, the Portuguese airports were found to be approximately
20\% more efficient than the Spanish ones.

The second-stage truncated regression provided insights into the exogenous factors influencing
Iberian airport efficiency. It demonstrated that airports located on islands, with military operations and
with a rail connection generally achieved higher efficiency scores. A higher share of cargo was also as-
sociated with increased efficiency, while the effect of the share of low-cost carrier (LCC) passengers was
inconclusive.

Lastly, the Malmquist Productivity Index (MPI) analysis showed that, over the period considered,
there was a modest improvement in MPI among the airports, with a value of 1.0412. This improvement
was primarily driven by efficiency change, while technological change had a smaller, but still positive,
impact.


\subsection{Public Policy Reflections}

The Iberian airport network, mainly due to the Spanish network, is characterized by a large number of underutilized airports, which negatively impacts the overall efficiency of
the system. This centralized management model in \textbf{Spain}, strongly relies on cross subsidies from large
airports to smaller ones, meaning that the profits from the larger airports are used to cover the losses
of the smaller ones. \cite{nerja2021} and \cite{cifuentes-faura2023} defended that this system further intensifies inefficiencies in the Spanish airport network. Conversely,
a different perspective defends that the actual system promotes equity and solidarity between Spanish
regions.

Prior studies, especially those published immediately before Aena’s privatization, identified persis-
tent sources of inefficiency and proposed different scenarios to mitigate them. The results obtained
in this dissertation invite policy makers and airport managers to reassess the current state of network
management model, taking those earlier proposals into account, namely, \textbf{regionalize} the management of underperforming airports;  \textbf{reallocate} cargo traffic from airports operating under decreasing returns to
scale (DRS) to those operating under increasing returns to scale (IRS); and consider  \textbf{closing or merging} airports that share catchment areas \cite{martin2001}, \cite{martin2011}, \cite{ripoll-zarraga2020}, \cite{nerja2021}, \cite{cifuentes-faura2023}.

\textbf{Portugal} has a much smaller airport network, especially on the mainland, so the above suggestions
for Spain are not directly transferable, since they can be more relevant for smaller to medium airports.
However, one main reflection should be done. Portugal's network is centered around 3 main airports, Lisbon, Porto and Faro, but none of them has an intermodal rail connection inside
the terminal. In Spain, by contrast, the largest mainland airports such as Madrid Barajas, Barcelona El
Prat, and Malaga, as well as a medium sized airport like Jerez with a direct link to Seville, do have rail
inside the terminal. The absence of rail connections in the main Portuguese airports limits the catchment
area of those airports, and consequently their potential passengers, efficiency and growth \cite{fernandez2022}. One can argue that the central train stations of those cities are relatively close to the airports and
only one transfer away. However, this indirect connectivity introduces inconveniences that not only increase travel time but also reshape passenger behaviour. Passengers relying on indirect connections,
driven by the higher degree of uncertainty, tend to arrive earlier at the airport to avoid the risk of delays
during the transfer, which can contribute for a higher concentration of passengers and consequently the
need for larger terminal facilities \cite{jiang2021}. Moreover,
the inconvenience of indirect connections can discourage passengers from using rail or public transport entirely, particularly when traveling with luggage. Instead, passengers may opt for private cars, which requires better and larger parking infrastructures \cite{pasha2020}.

The plan for the implementation of the new high-speed rail (HSR) line in Portugal, already contem-
plates the integration of Lisbon and Porto airports with future HSR stations (Infraestruturas de Portugal,
2025). The present study serves as evidence of the importance of this type of intermodal connections
for airport efficiency, and supports the decision of including those connections in the HSR Portugal’s
project.

Portugal operates its 10 main airports through ANA - Aeroportos de Portugal, a private company
owned by VINCI Airports, while the Spanish airport network comprises 46 airports managed by Aena, a
publicly owned company. Even though the Spanish airport network is much larger than Portugal’s, due
to the countries proximity and as both operate under centralized management models, it is important to
continuously benchmark the performance of Portuguese airports against the Spanish ones and follow
the evolution of the Spanish network closely.

Portugal’s concession agreement includes a 75 km exclusivity radius, within which new airports
cannot be developed without ANA’s consent. Outside this radius, airports may be developed indepen-
dently of the concession framework.
This specific context makes Castellón Airport a particularly relevant case study. Located about one
hour north of Valencia, approximately 100 km away, Castellón airport is managed by the Valencian
regional government outside the Aena network. Monitoring cases like Castellón Airport can provide insights
into how airports outside the centralized network perform and how they affect
the efficiency of airports in the network, as certain developments can be analogous to the Portuguese
context.

\subsection{Main Limitations and Future Work}
As any research work, this dissertation has some limitations that should be acknowledged by the
readers. Firstly, the DEA methodology, being a measure of relative efficiency, depends on the sample
of DMUs under analysis. Also, the choice of inputs and outputs can significantly influence the results,
since the efficiency scores are based only on the selected variables. Lack of availability of financial
data poses one of the main limitations of this study, since only technical efficiency could be assessed. For the Portuguese airports, only data regarding the 5 of the 10 ANA airports was
available, which limits the conclusions that can be drawn about the efficiency of the entire Portuguese
airport network. Regarding the explanatory variables, the share of LCC passengers was limited to the LCCs ranked
among the top 10 airlines at each airport, which may not fully capture the impact of LCCs on airport
efficiency.

Building on the present work, the following suggestions are made: As DEA measures relative efficiency, increasing the sample size by including more airports, especially the remaining Portuguese airports managed by ANA, would improve the robustness of the results and provide a wider benchmark comparison. Additionally, future Iberian studies could aim to include financial data, to assess cost
efficiency in addition to technical efficiency. As a continuation of the present work, the same methodology can be applied for the subsequent
years, i.e, from 2024 onwards.