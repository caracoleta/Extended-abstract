%%%%%%%%%%%%%%%%%%%%%%%%%%%%%%%%%%%%%%%%%%%%%%%%%%%%%%%%%%%%%%%%%%%%%%
%     File: ExtendedAbstract_intro.tex                               %
%     Tex Master: ExtendedAbstract.tex                               %
%                                                                    %
%     Author: Andre Calado Marta                                     %
%     Last modified : 27 Dez 2011                                    %
%%%%%%%%%%%%%%%%%%%%%%%%%%%%%%%%%%%%%%%%%%%%%%%%%%%%%%%%%%%%%%%%%%%%%%
% State the objectives of the work and provide an adequate background,
% avoiding a detailed literature survey or a summary of the results.
%%%%%%%%%%%%%%%%%%%%%%%%%%%%%%%%%%%%%%%%%%%%%%%%%%%%%%%%%%%%%%%%%%%%%%

\section{Introduction}
\label{sec:intro}

The air transport market has suffered significant transformations over the last decades, driven by the
liberalization of the industry, which started in 1978 when the United States of America (USA) deregulated
its domestic airline market, meaning airlines were free to set their own fares and routes, without federal
control. This new regulatory framework without market entry
barriers and market restrictions, marked the beginning of an era of unprecedented air traffic growth, where the number of passengers in the world increased from 310 million in 1970 to 4.5 billion in 2019 \cite{worldevolution}.

Central to this evolution was the emergence and expansion of Low-Cost Carriers (LCCs). Around 2002, a steep growth of LCCs started, reaching
30.3\% of the market share in 2018, 10 times more than in 2002 \cite{sharelcc}. Initially operating mostly from secondary or regional airports,
LCCs gradually expanded their operations to larger airports, and are now competing directly with legacy
airlines at major airports \cite{jimenez2020}. This strategic shift of LCCs towards larger airports highlights
the dynamic nature of the air transport market and the consequent need of constant benchmarking to evaluate the performance of airports and adapt accordingly to the changing market conditions. This need for performance analysis is further intensified by the increasing trend towards business
models such as privatization and the use of concessions for airport management, as regulation by public
entities and stakeholders requires transparent and objective performance metrics.

Benchmarking has become a fundamental tool to evaluate the performance across a wide range of
industries and represents a systematic comparison of the performance of different entities, which usually
transform the same kind of inputs into similar outputs \cite{bogetoft2011}.
 There are two main benchmarking methodologies: Stochastic Frontier Analysis (SFA), a parametric approach, meaning it requires a predefined functional form
for the production function, and the Data Envelopment Analysis (DEA), a non-parametric approach.. A thorough review of the existing literature on airport benchmarking was conducted to
identify the most common methodology and employed variables in this type
of analysis.

Motivated by the decision of the development of a new airport in Lisbon, the present work aims to benchmark the efficiency of 41 Iberian airports (5 Portuguese and 36 Spanish) from 2016 to
2023 and identify exogenous factors that influence airport efficiency, such as rail intermodal terminals.

In the next section, a literature review is presented. Then, the methodology used in this work is described, followed by the presentation and discussion of the results. Finally, the main conclusions are drawn and public policy reflections are provided.